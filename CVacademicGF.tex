
\documentclass[11pt,a4paper]{article}
%\usepackage[scaled]{helvet}
%\renewcommand\familydefault{\sfdefault}
% Palatino for rm and math | Helvetica for ss | Courier for tt
\usepackage{mathpazo} % math & rm
\linespread{1.05}        % Palatino needs more leading (space between lines)
\usepackage[scaled]{helvet} % ss
\usepackage{courier} % tt
\normalfont
\usepackage[T1]{fontenc}
\usepackage{fullpage}
\usepackage[latin1]{inputenc}
\usepackage{geometry}
\geometry{head=1.5\baselineskip, headsep=\baselineskip,textwidth=18cm,textheight=26cm}
%\textheight=9.0in
\raggedbottom
%\raggedright
\usepackage{lastpage}
\usepackage{multirow}
\setlength{\tabcolsep}{0in}
\setlength{\parindent}{0mm}
%\usepackage{lmodern}
\usepackage{setspace}
\usepackage{color}
\usepackage[usenames,dvipsnames,svgnames,table]{xcolor}
\onehalfspacing

\begin{document}

\makeatletter
 \renewcommand{\@evenhead}{\textit{G. F. Lima de Carvalho e Silva CV continued...}\hfil}
 \renewcommand{\@oddhead}{\@evenhead}
 \renewcommand{\@evenfoot}{\hfil \thepage\ of\  \pageref{LastPage}\hfil}
 \renewcommand{\@oddfoot}{\@evenfoot}
\makeatother



\thispagestyle{empty}
\begin{center}
{\huge\textbf{\textsc{Guilherme Frederico \\[3mm] Lima de Carvalho e Silva}}} \\ \vspace{0.5cm}
{\color{darkgray}1500 North Kuhn Drive  \\ Palmer, Alaska, 99645 \\
  guilherme.frederico.lima@gmail.com \\ +1 (907) 414 7973}
\end{center}
\vspace{0.5cm}
%\vspace{\baselineskip}
%\begin{tabular*}{18cm}{l@{\extracolsep{1mm}}l@{\extracolsep{\fill}}l@{\extracolsep{1mm}}l}
%\textbf{Address:} &Christ's College &\textbf{Email:} &guilherme.frederico.lima@gmail.com \\[1mm]
%&St. Andrews Street & \textbf{Tel:} &+44 (0) 7952 864 963\\[1mm]
%&Cambridge CB2 3BU & & \\[3mm]
%\textbf{Nationality:}& Brazilian &\textbf{Date of Birth:}& 24/01/1981
%\end{tabular*}
%\vspace{\baselineskip}


%\vspace{-0.2cm}

%\subsection*{Current and Former Positions}
%\begin{tabbing}
%Apr 2002 -- June 2006 \=\kill
%Oct 2013 -- present\> \textbf{College Lecturer in Mathematics} at Newnham College, and\\
%\>\textbf{Affiliated Lecturer} at Department of Pure Mathematics and\\
%\> Mathematical Statistics (DPMMS), University of Cambridge\\[.5\baselineskip]
%Jan 2011 -- Sep 2013 \> \textbf{College Lecturer in Mathematics} at Queens' College,\\
%\> University of Cambridge\\[.5\baselineskip]
%Jan 2010 -- Dec 2010 \> \textbf{Postdoctoral Researcher} at Universit\a'e catholique de Louvain\\
%\> funded by Fonds National de la Recherche Scientifique (FNRS), Belgium
%\end{tabbing}
%\vspace{-.7cm}
\begin{tabbing} 
Apr 2002 -- June 2006 \=\kill
\> \textbf{\Large{\textsc{Research Interests}}}\\
\> \noindent\rule{11cm}{1pt}\\
\\
\> Main \' Category Theory, Topos Theory, Grothendieck toposes, continuous \\ 
\> lattices, and topological aspects of Category Theory in general. \\
\\
\> Other \' Natural Language Processing, Cluster Analysis, applications of \\ 
\> NLP to psychiatry. 
\\
\end{tabbing}


\begin{tabbing} 
Apr 2002 -- June 2006 \=\kill
\> \textbf{\Large{\textsc{Education}}} \\
\> \noindent\rule{11cm}{1pt}\\
\> \\
\> Sept 2016 \' \textbf{PhD in Pure Mathematics}\\
\> Institution \' University of Cambridge, UK\\ 
\> Thesis title \' \textit{Cartesian and Finite-Product-Preserving} \\ \> \textit{Essential Inclusions of Grothendieck Toposes} \\
\>Supervisor \' Prof.\ Peter Johnstone.\\ [.5\baselineskip]

 \> Jun 2008 \' \textbf{Master of Mathematics}\\
\> Institution \' Pontif\a'icia Universidade Cat\a'olica do Rio de Janeiro - Brazil\\
\>Dissertation title  \' \textit{Sheaves and Topos Theory}\\
\>Supervisor \' Prof. George Svetlichny.\\ [.5\baselineskip]

\> Jul 2006 \' \textbf{Bachelor's degree in Pure Mathematics}\\
\>Pontif\a'icia Universidade Cat\a'olica do Rio de Janeiro - Brazil\\
\end{tabbing}




% \subsection*{Scholarships and Distinctions}
% \begin{tabbing}
% Apr 2002 -- June 2006 \=\kill
% Mar 2008 \>\textbf{Smith-Knight/Rayleigh-Knight Prize}\\
% \>Prize group 3; essay prize for 2nd year PhD students\\[.5\baselineskip]
% Oct 2006 -- Sep 2009 \>\textbf{Bachelor Scholarship}\\
% \> Awarded by Emmanuel College for first class performance in Part III\\[.5\baselineskip]
% Oct 2002 -- Jun 2006 \>\textbf{Cambridge European Trust Bursary}\\
% \>\pounds 1000 per year\\[.5\baselineskip]
% since Oct 2004 \>\textbf{Fellow of the Cambridge European Society}\\
% \> Awarded by Cambridge European Trust\\[.5\baselineskip]
% Oct 2004 -- Jun 2005 \>\textbf{Benson and Carlsaw Senior Scholarship, Braithwaite Batty Prize}\\
%  \>Awarded by Emmanuel College for first class performance in Part IB\\[.5\baselineskip]
% Apr 2002 -- Jun 2006 \>\textbf{Studienstiftung des Deutschen Volkes}\\ 
% \> (German National Academic Foundation)\\
% \> Highly selective scholarship offering regular academic programmes (summer schools)
% \end{tabbing}
%\clearpage
%\vspace{-.7cm}
\begin{tabbing}
Apr 2002 -- June 2006 \=\kill
\> \textbf{\Large{\textsc{Talks}}} \\
\> \noindent\rule{11cm}{1pt}\\
\> \\
\> July 2017 \' \textbf{International Category Theory Conference 2017}\\ 
\> University of British Columbia, Vancouver, Canada\\
\> Presentation: \textbf{\textit{Duality theorems for essential inclusions}} \\ \> \textbf{\textit{of Grothendieck toposes}}\\ [1mm] 
\> May 2016 \' \textbf{100th Peripatetic Seminar on Sheaves and Logic } \\
\> University of Cambridge, UK \\
\> Presentation: \textbf{\textit{Pullbacks of Essential Inclusions}} \\ \> \textbf{\textit{of Grothendieck Toposes}} \\[1mm]
\\
\> Nov 2015 \' \textbf{Topos \a`a l'IHES} \\ 
\> Institut des Haute \a'Etudes Scientifiques, Paris, France\\
\> Presentation: \textbf{\textit{From Essential Inclusions to Local}} \\ \> \textbf{\textit{Geometric Morphisms}}\\[1mm]
\> Jun 2015 \' \textbf{International Category Theory Conference 2015}\\ 
\> University of Aveiro, Portugal\\
\> Presentation: \textbf{\textit{Site characterisations for Local}} \\ \> \textbf{\textit{Geometric Morphisms}}\\ [1mm] 
\> Jul 2013 \' \textbf{International Category Theory Conference 2013}\\
\> Macquarie University, Sydney, Australia\\ 
\> Presentation: \textbf{\textit{Surjections of Grothendieck Toposes}}\\ [1mm]
\>Jun 2013 \' \textbf{Young Researchers in Mathematics 2013}\\
\> University of Edinburgh, UK\\
\>Presentation:
 \textbf{\textit{Exponentiable Toposes}}\\[1mm]
\>Apr 2012 \' \textbf{Young Researchers in Mathematics 2012}\\ 
\>Bristol University, UK \\
\>Presentation:
 \textbf{\textit{An Introduction to Classifying Toposes}}\\[1mm]
%Jul 2011\>CT 2011, Vancouver, Canada\\[1mm]
\> Apr 2011 \' \textbf{Young Researchers in Mathematics 2011}\\ 
\> Warwick University, UK\\
\>Presentation:
 \textbf{\textit{Topos Theory}}\\[1mm]
\end {tabbing}
\begin{tabbing} 
 Apr 2002 -- June 2006 \=\kill
\> \textbf{\Large{\textsc{Scholarships and Awards}}} \\
\> \noindent\rule{11cm}{1pt}\\
\> \\
\> 2016 \' Cambridge Philosophical Society Research Studentship, \\ \> to conclude promising piece of research \\ [1mm]
\> 2011, 2013 \' DPMMS travel grants for conferences in Vancouver \\ \>  and Sydney \\ [1mm]
\> 2010 \' Smith-Knight and Rayleigh-Knight essay competition 3rd prize \\ [1mm]
\> 2009 \' Cambridge Trusts International Scholarship \\ \> 3 year full PhD scholarship \\ [1mm]
\> 2007 \' Bolsa Aluno Nota Dez (Grade A Student Scholarship) \\ \> 2 year Masters scholarship \\
\end{tabbing}

\newpage 
\begin{tabbing}
Apr 2002 -- June 2006 \= Co \=\kill
\> \textbf{\Large{\textsc{Teaching Experience}}} \\
\> \noindent\rule{11cm}{1pt}\\
\> \\
%\>{\bf Metric and Topological Spaces} (second year), 2011\\[2mm]
%Oct 2008 \>\textbf{Induction Talk} on proof techniques for Emmanuel first-year mathematicians\\[1mm]
\> Jan 2010 -- Apr 2017  \' \textbf{Supervisor for Cambridge Part III courses}\\ 
\> Employed by different constituent colleges of the University of \\ \> Cambridge  to tutor individually or in classes for Masters level courses.\\
\> Courses: \\
\> \> $\bullet$ {\bf Category Theory}, since 2010 \\ 
\> \> $\bullet$ {\bf Topos Theory}, since 2012\\ [5mm]
\> Oct 2012 -- Apr 2017 \' \textbf{Supervisor for Cambridge Mathematical Tripos}\\
\> Employed by different constituent colleges of the University of \\ \> Cambridge to teach undergraduates in pairs, parallel to their lecture courses.\\
\> Courses: \\ 
\> \> $\bullet$ {\bf Numbers and Sets} (first year), 2015 \\
\> \> $\bullet$ {\bf Metric and Topological Spaces} (second year), since 2012\\
\> \> $\bullet$ {\bf Linear Algebra} (second year), 2016\\
\> \> $\bullet$ {\bf Groups, Rings and Modules} (second year), 2016\\
\> \> $\bullet$ {\bf Logic and Set Theory} (third year), 2013\\
\> Colleges: \\
\> \> Trinity College, Emmanuel College, Christ's College, \\ 
\> \> Churchill College, Queens' College, Newnham College,\\ 
\> \> Murray Edwards College.\\ [5mm]
\> Feb 2009 -- Jun 2009 \' \textbf{Calculus Lecturer} for Pontif\a'icia Universidade Cat\a'olica\\ \> do Rio de Janeiro, Brazil\\ 
\end{tabbing}
%\vspace{-.5cm}





%\clearpage


\begin{tabbing}
Apr 2002 -- June 2006 \=\kill
\> \textbf{\Large{\textsc{Seminar Organisation}}} \\
\> \noindent\rule{11cm}{1pt}\\
\> \\
\> Oct 2012 - Jan 2015 \' \textbf{Junior Category Theory Seminar} \\
\> Created and ran the weekly seminar, which was presented and attended  \\ \> by Masters and PhD students of the University of Cambridge.\\ [2mm] 
\> Oct 2012 - Mar 2014 \' \textbf{Joint Category Theory and Computer Science Seminar Series}\\ 
\> Created and organised the seminar series in 2012 and 2013. \\ 
\> Talks were given by PhD students of both Pure Maths \\ \> and Computer Science departments of the Univeristy of Cambridge.\\  
\end{tabbing}
%\vspace{-.5cm}

\newpage 
\begin{tabbing}
Apr 2002 -- June 2006 \=\kill
\> \textbf{\large{\textsc{Additional Activities}}} \\
\> \noindent\rule{11cm}{1pt}\\
\\
\> 2014  \' Created and assumed the role of president of the \textbf{Christ's College Volleyball} \\ 
\> \textbf{Club}, affiliated to the Christ's College Sports Society, in the University of \\ \> Cambridge. \\ [2mm]
\> 2010 \' Co-founded and assumed the role of vice-president in the year of 2010
of the \\ \> \textbf{Cambridge University Brazilian Society}, which organises academic \\ 
\>and cultural events promoting Brazilian culture in Cambridge. \\ [5mm]
\> \textbf{Computing:} Python, \LaTeX, Maple. \\ 
\> \textbf{Languages:} Portuguese (native), English (fluent), French (intermediate). \\  
\end{tabbing}
\vspace*{\fill}
\hspace{3.5cm} \textit{Updated July 2018}



%\subsection*{Skills}
%\begin{tabbing}
%Apr 2002 - June 2006 \=\kill
%\textbf{Computing} \> C, \LaTeX , Linux (KdE and Gnome), Word, PowerPoint; can touchtype\\
%\textbf{Languages} \>German (native), English (fluent), French (intermediate)\\
%\textbf{Presentation} \>Gave many talks and seminars at university and conferences\\
%\textbf{Organisation} \>Organised events for CUGMS, directed Part III Seminar series\\
%\textbf{Teamwork} \>CUGMS committee, Part III Seminars work with codirector and group leaders\\
%\textbf{Driving} \>German driving licence
%\end{tabbing}


%\begin{itemize}
%\item[-] some experience with C and \LaTeX\  through Computer Projects for my course at Cambridge and typesetting papers
%\item[-] good presentational skills through giving many talks, presentations and seminars at school, university and conferences
%\item[-] experience in organising, through directing Part III Seminar Series in Michaelmas and Lent Term 2007/2008
%\item[-] team work experience through my work at the Weilburger Schlosskonzerte and a Team Programing Project at the Computer Summer Camp in Passau
%\end{itemize}
\end{document}

%{\large \textbf{Skills}}

	
%	\item[]%
%	\begin{tabular*}{6in}{l@{\extracolsep{\fill}}r}
%	 \textbf{Gymnasium Philippinum Weilburg} & October 1992 - July 2001\\
%	 Highschool &\\
%	 \end{tabular*}
%\end{itemize}


%\begin{description}
%\item[Languages:]
%Python, C, Perl, Bourne shell scripting, BASH shell scripting, SQL, HTML, XML, CSS, XHTML, \LaTeX, Java, PHP, Javascript, Coldfusion, Scheme
%\item[Operating Systems:]
%Linux, Solaris, FreeBSD, OpenBSD, MacOS X, Windows 95/98/NT/2000
%\item[Applications:]
%Oracle, Apache, CVS, XFree86, MySQL, GTK
%\item[Miscellaneous:]
%system administration in various UNIX flavors, TCP/IP networking, software configuration management, strong verbal and written communication skills, excellent troubleshooting and debugging skills
%\end{description}